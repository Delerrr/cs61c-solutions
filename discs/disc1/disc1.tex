\documentclass[12pt,a4paper]{article}
\usepackage[utf8]{inputenc}
\usepackage[T1]{fontenc}
\usepackage{amsmath}
\usepackage{amssymb}
\usepackage{graphicx}
\usepackage{color}
\usepackage{ulem}
\title{disc1}
\begin{document}
\maketitle
	\begin{enumerate}
		\item 
		\begin{itemize}
			\item 1.1 True. 
			\item 1.2 False
			\item 1.3 False
			\item 1.4 True
		\end{itemize}
		\item 
		(a)
		\begin{itemize}
			\item 
				Binary:	\underline{0b10010011}\\
				Hex:	0x93\\
				Dec:	147
			\item 
				Binary:	0b01001111\\
				Hex:	\sout{0x39}(\textcolor{red}{0x3F})\\
				Dec:	\underline{63}
			\item 
				Binary:	\underline{0b00100100}\\	
				Hex:	0x24\\
				Dec:	36
			\item 
				Binary:	0b0\\
				Hex:	0x0\\
				Dec:	\underline{0}
			\item 
				Binary:	0b00100111\\
				Hex:	0x27\\
				Dec:	\underline{39}
			\item 
				Binary:	0b0000000110110101\\
				Hex:	0x01B5\\
				Dec:	\underline{437}
			\item 
				Binary:	0b0000000100100011\\
				Hex:	\underline{0x0123}\\
				Dec:	291
		\end{itemize}
	(b)	
		\begin{itemize}
			\item 0b1101 0011 1010 1101
			\item 0b1011 0011 0011 1111
			\item 0b0111 1110 1100 0100  
		\end{itemize}
	(c)\\
		64Ki	128Mi	8Ti		64Gi\\
		\sout{8Ti}(\textcolor{red}{16Gi})		2Ei		128Ti	512Pi\\
	(d)\\
		$ 2^{\sout{12}(\textcolor{red}{11})} $	$ 2^{19} $	$ 2^{24} $\\
		$ 2^{58} $	$ 2^{36} $	$ 2^{67} $
		
	\item (a)\\
	Unsigned:	0xFF(255)	0x0(0)\\
	Biased:		0xFF(128)	0x0(-127)\\
	Two's:		0x7F(127)	0x80(-128)\\
	(b)\\
	Unsigned:	0x0		0x1		No way\\
	Biased:		0x7F	0x80	0x7E\\
	Tow's:		0x0		0x1		0xFF\\
	(c)\\
	Unsigned:	0x11	No way\\
	Biased:		0x90	0x6E\\
	Two's:		0x11	0xEF\\
	(d) \sout{127}(\textcolor{red}{There is no such integer. For example, an arbitrary 8-bit mapping could choose
		to represent the numbers from 1 to 256 instead of 0 to 255.})\\
	(e)...\\
	(f) I don't know.\\\textcolor{red}{Decimal is the preferred radix for human hand calculations, likely related to
		the fact that humans have 10 fingers.\\\\
		Binary numerals are particularly useful for computers. Binary signals are
		less likely to be garbled than higher radix signals, as there is more “distance”
		(voltage or current) between valid signals. Additionally, binary signals are quite
		convenient to design circuits, as we’ll see later in the course.\\\\
		Hexadecimal numbers are a convenient shorthand for displaying binary numbers,
		owing to the fact that one hex digit corresponds exactly to four binary digits.}
	\item 
		\begin{enumerate}
		\item 
			\begin{itemize}
			\item 0b010010(18)	No overflow
			\item 0b011101(29)	Overflow
			\item 0b000001(1)	\textcolor{red}{Not }Overflow \textcolor{red}{(-5 + 6 = 1)}
			\item Impossible(0xAA is at least 8 bits)
			\end{itemize}
		\item 
			\begin{itemize}
			\item 9
			\item 6
			\item \sout{6}(\textcolor{red}{7})
			\item 44
			\end{itemize}
		\end{enumerate}
	\end{enumerate}
\end{document}